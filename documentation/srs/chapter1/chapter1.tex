\section{Introduction}

	\subsection{Purpose}
		The purpose of this document is to define the requirements for Bitcoin WalleTx. The intended audience of this document includes the CEN4021 instructional staff at FSU and the end users of the WalleTx. It is also intended that this document serve as the principle point of reference for the Prestige Worldwide team throughout the development of WalleTx.\\

	Bitcoin WalleTx (WalleTx) is a bitcoin wallet tracker tool for Android that assists users in monitoring their bitcoin balance, transaction history, and spending trends across multiple wallets. It is common for bitcoin users to possess numerous wallets, yet there do not exist many services that are capable of aggregating wallet information in order to provide the user with an overall picture of their bitcoin funds. Bitcoin WalleTx solves this problem by allowing users to group or categorize their bitcoin wallets, as well as tag their transactions with real-world information related to how their bitcoins are being spent. Charts and graphs help to identify trends in both single wallet spending and across wallet groups, a useful feature that is essential for enabling users to integrate their bitcoin finances with a more traditional budget.\\

	\subsection{Scope}
	
  \subsubsection{Product}
  Bitcoin WalleTx (referred to as WalleTx)

  \subsubsection{Scope of Work}
  WalleTx is an android application that enables users to import their bitcoin wallets by public key and tag their bitcoin transactions in order to obtain an aggregated overview of their bitcoin finances. WalleTx shall allow users to manage their public keys; manage wallet groups; manage transaction tags; tag individual transactions; and receive feedback regarding trends via charts and graphs. A full listing of functional requirements for WalleTx is located in the Functional Requirements section of this document.
	
  \subsubsection{Out of Scope}
  WalleTx shall not provide any functionality allowing users to spend or receive bitcoins since only public keys shall be stored on the device of the user.
	
  \subsubsection{Application} 
  Bitcoin is a new technology and the ecosystem is currently undergoing major infrastructure development. There currently do not exist any tools that allow users to label their bitcoin transactions in order to identify trends in their bitcoin spending behavior. WalleTx aims to bring this functionality to the Bitcoin space.
    
	\subsection{Definitions}

	\begin{itemize}
    \item \textbf{WalleTx} - Shorthand for Bitcoin WalleTx
    \item \textbf{Bitcoin} - a tradeable virtual asset existing on the ledgers of a distributed and synchronized network of ledgers
    \item \textbf{bitcoin} - a tradeable virtual asset unit, divisible into 100 million units (also known as a Satoshi) 
    \item \textbf{Blockchain} - A blockchain is made of a blocks, linked to the block before and block after, that contain information pertinent to the underlying system. In the case of Bitcoin these blocks contain transaction information.
    \item \textbf{Public key} - A hash of a wallet's public key that allows other users on the Bitcoin network to send bitcoins to the Public Key. 
    \item \textbf{Tx} - A transaction that moves bitcoins from one address to another, usually in exchange for goods or services.
	\end{itemize}

	\subsection{Acronyms}

	\begin{description}
		\item[BTC] Common unit of Bitcoin currency
	\end{description}

	\subsection{References}

	\begin{itemize}
		\item Bitcoin: A Peer-to-Peer Electronic Cash System\\ \url{https://bitcoin.org/bitcoin.pdf}
		\item Blockchain Data API\\ \url{https://blockchain.info/api/blockchain_api}
	\end{itemize}

	\subsection{Overview}

  Description of the remainder of the SRS.
  The remainder of the document outlines the functionality and operation of the WalleTx software. It outlines a general description of who, what, why, when, and how individuals will use our software. It provides the functional and non-functional constraints our software will incur. The document will also serve to illustrate software functionality from a high level architectural view to minute details about every user scenario the software may encounter. The appendix provides information and graphs pertinent to many of these details.
