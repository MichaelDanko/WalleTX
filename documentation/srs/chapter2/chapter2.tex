\section{General Description}
  WalleTx is an internet connected, bitcoin connected software that is used as a financial managemnt tool. With this software you are able to review a history of spending with the option to categorize individual transactions. 
  \subsection{Product Perspective}
  WalleTx is dependent on the bitcoin blockchain to source information regarding tranactions amounts, times, and dates. It is possible that manual data entry can be implemented for those transactions however it is assumed the blockchain access will be available at all times. If there is not internet, you cannot spend your bitcoin. Accessing this software will be completed through mobile smartphones with Android operating systems.
  \subsection{Product Functions}
  Product functions include accessing the bitcoin blockchain to download transactions, allowing users to tag and categoryize transactions into which categories they wish. And then to analyze data by association to come to their own conclusions with regard to spending habits. Users will be able to add multiple wallets to allow for spending and saving 'wallets'. 
  \subsection{User Characteristics}
  Users are from a broad spectrum of every day life. It is WalleTx's intention to be available to people of all ages from every corner of the globe. Bitcoin is a global network, with many users from young to old and WalleTx will be usuable by the same demographic.  Anyone who is capable of utilizing bitcoin technology to transact will be able to use WalleTx to manage their financial records. 
  \subsection{General Constraints}
  The system will be designed by a group of 4 junior developers and this may impose constraints on development. The software will not be constrained by regulatory policies.

  To finish

  (1)  Regulatory policies

  (2)  Hardware limitations; for example, signal timing requirements

  (3)  Interface to other applications

  (4)  Parallel operation

  (5)  Audit functions

  (6)  Control functions

  (7)  Higher-order language requirements

  (8)  Signal handshake protocols; for example, XON-XOFF, ACK-NACK.

  (9)  Criticality of the application

  (10) Safety and security considerations
  \subsection{Assumptions and Dependencies}

  The assumption is that the application will be used on a mobile device. The next assumption is that the mobile device is utilizing the Android operating system. At this time the application will only be able to run on the Android operating system.
